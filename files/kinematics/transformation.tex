\documentclass[12pt,a4paper]{article}

%\documentclass[12pt,twoside,openright]{report}
\usepackage{graphicx} 	 	
\usepackage{listings} 	 	%
\usepackage{amsmath} 
\usepackage{amssymb}
\usepackage{float}
\usepackage{mathtools}
%\usepackage[Bjornstrup]{fncychap}
\usepackage{subcaption}
\usepackage{pdfpages}
\usepackage{multirow}

\begin{document}

\section{Transformations}
\label{triped_trafo}

\subsection{sub sub chain 0 of the closed sub chain}
The open sub sub chain 0 of the closed chain:
\begin{equation}
    A_{\text{Chain 0}} = A_{P}^{CCS}
\end{equation}
\begin{equation}
    A_{P}^{CCS}
    =
    \begin{pmatrix}
     - 2*c_0^2 - 2*d_0^2 + 1 &     2*b_0*c_0 - 2*a_0*d_0 & 2*a_0*c_0 + 2*b_0*d_0&  0.265 \\
      2*a_0*d_0 + 2*b_0*c_0 & - 2*b_0^2 - 2*d_0^2 + 1 &     2*c_0*d_0 - 2*a_0*b_0 &      0\\
      2*b_0*d_0 - 2*a_0*c_0 &     2*a_0*b_0 + 2*c_0*d_0 & - 2*b_0^2 - 2*c_0^2 + 1 &  0.014 \\
                      0 &                     0 &                     0&      1
    \end{pmatrix}
\end{equation}
with $a_0,b_0,c_0$ and $d_0$ as the quaternion variables of the LCS joint. \\


\subsection{sub sub chain 1 of the closed sub chain}
The open sub sub chain 1 of the closed sub chain, is made up as follows:
\begin{equation}
    A_{\text{Chain 1}} = A_{MCS1}^{CCS} * A_{Sph-1-1}^{MCS1} * A_{Sph-1-2}^{Sph-1-1} * A_{P}^{Sph-1-2}  
\end{equation}

The transformation matrix $A_{MCS1}{CCS}$ is made up of 2 transformations. A static translation and rotation from the ccs to the base frame of the mcs1 joint $A_{MCS1-Base}^{CCS}$ and a rotation around the z axis of the base frame of the joint $A_{\text{MCS1 Joint}}$.
\begin{equation}
    A_{MCS1}{CCS} = A_{MCS1-Base}^{CCS} * A_{\text{MCS1 Joint}}
\end{equation}

\begin{equation}
    A_{CCS}^{MCS1} = 
    \begin{pmatrix}
    0.9306   & -0.3661  &       0   & 0.139807669447128 \\
    0.3661   & 0.9306   &        0  & 0.0549998406976098\\
    0        &     0    &  1.0000   & -0.0510\\
    0       &      0    &      0    & 1.0000\\
    \end{pmatrix}
\end{equation}

\begin{equation}
    A_{\text{MCS1 Joint}} = 
    \begin{pmatrix}
        cos(\theta_1) & -sin(\theta_1)& 0 & 0\\
        sin(\theta_1)&  cos(\theta_1)& 0 & 0 \\
        0 & 0 & 1 & 0 \\
        0 & 0 & 0 & 1
    \end{pmatrix}
\end{equation}

The transformation $A_{Sph-1-1}^{MCS1}$ consists of a translation along the output lever and a rotation around the spherical joint 1-1:
\begin{equation}
        A_{Sph-1-1}^{MCS1} =
        \begin{pmatrix}
        R_{quat}(a_{1,1},b_{1,1},c_{1,1},d_{1,1}) & \begin{pmatrix} 0.085 \\0 \\-0.0245 \end{pmatrix}\\
        (0, 0, 0)& 1
        \end{pmatrix}
\end{equation}
%TODO in simulation translation value is [85 0 (-17.5 -7)]
With $a_{1,1},b_{1,1},c_{1,1},d_{1,1}$ as the quaternion variables of the spherical-1-1 joint and $R_quat$.\\

Next is the transformation $A_{Sph-1-2}^{Sph-1-1}$, which includes a translation along the coupling rod and a rotation around the spherical joint 1-2:
\begin{equation}
    A_{Sph-1-2}^{Sph-1-1}
    =
    \begin{pmatrix}
        R_{quat}(a_{1,2},b_{1,2},c_{1,2},d_{1,2}) & \begin{pmatrix} 0.11 \\0 \\0 \end{pmatrix}\\
        (0, 0, 0)& 1
        \end{pmatrix}
\end{equation}
Last is a translation from the spherical joint 1-2 to the $P$ frame.
\begin{equation}
    A_{P}^{Sph-1-2}
    =
    \begin{pmatrix}
        \begin{pmatrix}
        1&0&0\\
        0&1&0 \\
        0&0&1\\
        \end{pmatrix} 
        & 
        \begin{pmatrix} -0.015 \\-0.029 \\0.0965 \end{pmatrix}\\
        (0, 0, 0)& 1
        \end{pmatrix}
\end{equation}

\subsection{sub sub chain 2 of the closed sub chain}
The transformation matrix for the open sub sub chain 2 of the closed chain can be calculated as follows: \\
The open sub sub chain 2 of the closed sub chain, is made up as follows:
\begin{equation}
    A_{\text{Chain 2}} = A_{MCS2}^{CCS} * A_{Sph-2-1}^{MCS2} * A_{Sph-2-2}^{Sph-2-1} * A_{P}^{Sph-2-2}  
\end{equation}

The transformation matrix $A_{MCS2}{CCS}$ is made up of 2 transformations. A static translation and rotation from the ccs to the base frame of the mcs2 joint $A_{MCS2-Base}^{CCS}$ and a rotation around the z axis of the base frame of the joint $A_{\text{MCS2 Joint}}$.
\begin{equation}
    A_{MCS2}{CCS} = A_{MCS2-Base}^{CCS} * A_{\text{MCS2 Joint}}
\end{equation}

\begin{equation}
    A_{CCS}^{MCS2} = 
    \begin{pmatrix}
    0.9306   & 0.3661  &       0   & 0.139807669447128 \\
    -0.3661   & 0.9306   &        0  & -0.0549998406976098\\
    0        &     0    &  1.0000   & -0.0510\\
    0       &      0    &      0    & 1.0000\\
    \end{pmatrix}
\end{equation}

\begin{equation}
    A_{\text{MCS2 Joint}} = 
    \begin{pmatrix}
        cos(\theta_2) & -sin(\theta_2)& 0 & 0\\
        sin(\theta_2)&  cos(\theta_2)& 0 & 0 \\
        0 & 0 & 1 & 0 \\
        0 & 0 & 0 & 1
    \end{pmatrix}
\end{equation}

The transformation $A_{Sph-2-1}^{MCS2}$ consists of a translation along the output lever and a rotation around the spherical joint 2-1:
\begin{equation}
        A_{Sph-2-1}^{MCS2} =
        \begin{pmatrix}
        R_{quat}(a_{2,1},b_{2,1},c_{2,1},d_{2,1}) & \begin{pmatrix} 0.085 \\0 \\-0.0245 \end{pmatrix}\\
        (0, 0, 0)& 1
        \end{pmatrix}
\end{equation}
With $a_{2,1},b_{2,1},c_{2,1},d_{2,1}$ as the quaternion variables of the spherical-2-1 joint and $R_quat$ as specified in \eqref{eq:quaternion2RotMat}.\\

Next is the transformation $A_{Sph-2-2}^{Sph-2-1}$, which includes a translation along the coupling rod and a rotation around the spherical joint 2-2:
\begin{equation}
    A_{Sph-2-2}^{Sph-2-1}
    =
    \begin{pmatrix}
        R_{quat}(a_{2,2},b_{2,2},c_{2,2},d_{2,2}) & \begin{pmatrix} 0.11 \\0 \\0 \end{pmatrix}\\
        (0, 0, 0)& 1
        \end{pmatrix}
\end{equation}
Last is a translation from the spherical joint 2-2 to the $P$ frame.
\begin{equation}
    A_{P}^{Sph-2-2}
    =
    \begin{pmatrix}
        \begin{pmatrix}
        1&0&0\\
        0&1&0 \\
        0&0&1\\
        \end{pmatrix} 
        & 
        \begin{pmatrix} -0.015 \\0.029 \\0.0965 \end{pmatrix}\\
        (0, 0, 0)& 1
        \end{pmatrix}
\end{equation}

\subsection{The open sub chain}
The open sub chain $A_{\text{Open Chain}}$ of the hybrid chain consists of 2 transformations:
\begin{equation}
    A_{\text{Open Chain}} = A_{LL-Joint}^{P} * A_{FCS}^{LL-Joint} 
\end{equation}
The $A_{LL-Joint}^{P}$ matrix includes a transformation to the base frame of the LL joint and a rotation around the base frames y axis to the follower frame of the LL joint.
\begin{equation}
     A_{LL-Joint}^{P} = 
     \begin{pmatrix}
     cos(\theta_{LL}) & 0 & sin(\theta_{LL}) & 1.640\\
     0 & 1 & 0 & 0\\
     -sin(\theta_{LL}) & 0 & cos(\theta_{LL}) & -0.037
     \end{pmatrix}
\end{equation}
With $\theta_{LL}$ as the angle in $radian$, which describes the rotation around the y axis of the LL base frame. This angle can be calculated from the angle of the leg extend motor joint $\theta_0$, as follows:
\begin{equation}
    \theta_{LL} = -1* \theta_0 * 0\frac{0.07}{2\pi1.5} + startingValue
\end{equation}
The $startingValue$ is an offset for the zero position of the joint rotation. Usually the $startingValue = 0 (rad)$, however it is possible to change it, to fit different zero rotation states of the leg, e.g if the zero state should be, when the light barrier sensor triggers. \\

Finally the $ A_{FCS}^{LL-Joint}$ transformation is a simple translation along the x axis of the LL joint base frame.

\begin{equation}
    A_{FCS}^{LL-Joint} = 
    \begin{pmatrix}
    1 & 0 & 0 & -1.5\\
    0 & 1 & 0 & 0\\
    0 & 0 & 1 & 0\\
    0 & 0 & 0 & 1\\
    \end{pmatrix}
\end{equation}

\end{document}